% Created 2020-09-20 Sun 18:41
% Intended LaTeX compiler: pdflatex
\documentclass[11pt]{article}
\usepackage[utf8]{inputenc}
\usepackage[T1]{fontenc}
\usepackage{graphicx}
\usepackage{grffile}
\usepackage{longtable}
\usepackage{wrapfig}
\usepackage{rotating}
\usepackage[normalem]{ulem}
\usepackage{amsmath}
\usepackage{textcomp}
\usepackage{amssymb}
\usepackage{capt-of}
\usepackage{hyperref}
\usepackage{minted}
\usepackage[, germanb]{babel}
\author{Jakob Klemm}
\date{\today}
\title{Simulation}
\hypersetup{
 pdfauthor={Jakob Klemm},
 pdftitle={Simulation},
 pdfkeywords={},
 pdfsubject={},
 pdfcreator={Emacs 27.1 (Org mode 9.3)},
 pdflang={Germanb}}
\begin{document}

\maketitle
\tableofcontents

\section{Einführung}
\label{sec:org278b65d}
Auch wenn sich die Flugbahn einer \texttt{Saturn V} gut in \texttt{Excel} oder einem ähnlichen
Programm berechnen lässt, so sind diese Programme doch limitiert. Sowohl in
ihrer Geschwindigkeit, als auch in den Funktionen und Einstellungen, welche für
eine solche Simulation relevant sein könnten. Mit den wenigen Werten, welche uns
für diese Simulation zur Verfügung standen, war es schnell möglich, neue Werte
und Flugbahnen zu berechnen.

Und trotzdem ist es äusserst kompliziert, die Simulation interaktiv zu gestalten
und dem Betrachter die Möglichkeit zu geben, selber die Kontrolle zu übernehmen.
Dazu kommt ebenfalls noch die starke Systemgebundenheit solcher Dateien. Auch
wenn \texttt{Excel} weit verbreitet ist und es genügend kostenlose Alternativen gibt,
welche in der Lage sind, \texttt{Excel-Dateien} zu öffnen, ist man doch an die Funktionen
und Bedingungen dieser Programme gebunden. Obwohl einige \emph{kostenlos} sind, findet
man kaum \emph{freie} oder gar \emph{open-source} Varianten.

Die Formeln und Konzepte, mit welchen die Geschwindigkeiten und Flugbahnen einer
\texttt{Saturn V} berechnet werden, lassen sich unter verschiedenen Bedienungen und für
verschiedene Raketen oder Flugobjekte anwenden. Um den Prozess interaktiver und
intuitiver zu machen, war es von Nöten, die Simulation als ein ausführbares,
interaktives Programm zu implementieren.
\section{Implementierung}
\label{sec:orge7180a8}
Den Begriff \emph{Raketensimulator} verbinden die meisten Menschen im ersten Moment mit
Supercomputern und hoch komplexen Berechnungen. Trotzdem lässt sich das meiste
dennoch auf einfache, lineare oder exponentielle Gleichungen bringen, die von
Hand oder mit einem einfachen Taschenrechner gelöst werden können.

Daher war auch die Wahl der Programmiersprache eher offen, da die Berechnungen
selbst nicht besonders aufwendig sind. Trotzdem gab es einige Kriterien, welche
eine Programmiersprache für diese Aufgabe erfüllen musste:
\begin{itemize}
\item Die Fähigkeit Dateien zu lesen, zu interpretieren und zu schreiben.
\item Einfache, lineare Algebra sowie häufige mathematische Funktionen.
\item Zugriff auf externe Programme über eine standardisierte Schnittstelle.
\item Kompilierung in eine alleinstehende, ausführbare Datei.
\end{itemize}

Auch wenn viele Sprachen und Programme zur Verfügung standen, fiel die Wahl auf
\texttt{Golang}, eine Sprache die ursprünglich von \texttt{Google} entwickelt wurde. \texttt{Golang}
erlaubt Entwickler schnelle, parallele und nebenläufige Programme systemnah zu
schreiben.
\subsection{Struktur}
\label{sec:orgc35e631}
Obwohl der gesamte Code \emph{open-source} und auf \texttt{Github} verfügbar ist, sollen einige
Funktionen hier trotzdem besprochen werden, da die Codebase verwirrend sein
kann.
\subsubsection{Menü}
\label{sec:org18d02c2}
Das Menü ist technisch zwar der am wenigsten interessante Teil, doch es ist ein
äusserst relevantes Stück der Simulation. Basierend auf einfachen
Konsolen-Printouts und Zahleninputs, macht dieses Menü die einfache Interaktion
mit der Software möglich, ohne dabei besondere Anleitungen oder Erklärungen zu
verfassen.
\subsubsection{Einstellungen}
\label{sec:orgd3cd208}
Da Raketen unter verschiedenen Kriterien und in verschiedenen Situationen
simuliert werden können, war es wichtig, die einzelnen Funktionen und Parameter
sowohl einstellbar, als auch ausschaltbar zu machen. Dies geschieht über
\texttt{settings.json}, einem einfachen \texttt{JSON-file}, worin alle Parameter und
Einstellungsmöglichkeiten, sowie eine Liste der verfügbaren Umgebungen
aufgelistet sind.\\
Über das Hauptmenü und die beiden Untermenüs ist es möglich, die Kriterien für
den Flug zu bestimmen. Sobald diese feststehen, kann die Simulation gestartet
werden.
\subsubsection{Module}
\label{sec:org216e70c}
Da verschiedene Raketen verschiedene Treibstoffe, Bauarten, Triebwerke und
Flugbahnen besitzen, war es ebenfalls wichtig, eine Software zu schreiben, die
für verschiedene Situationen gewappnet ist. Im dedizierten \texttt{rockets/} Ordner
können beliebige Konfigurationen und Modelle in Form von \texttt{JSON-Dateien} abgelegt
werden, welche dann der Simulation zur Verfügung stehen.
\subsection{Berechnungen}
\label{sec:org3fa98a3}
Auch wenn die \texttt{Raketengleichung} und andere Hilfsmittel zur Verfügung stehen, ist
es doch weiterhin möglich, die Raketen schrittweise zu simulieren. Da jeder
Schritt in einer solchen Simulation beinahe identisch abläuft, aber trotzdem von
allen vorherigen Schritten abhängig ist, sowie alle nachfolgenden Schritte
beeinflusst, war es nur logisch, die Simulation rekursiv zu implementieren. \\

Während das Berechnen der neuen Masse eine einfache Rechnung darstellte, mussten
kompliziertere mathematische Konzepte verwendet werden, um die neue
Geschwindigkeit zu berechnen. Zuerst wird über die Impulserhaltung die neue
Geschwindigkeit ohne externe Kräfte, hier \texttt{ideal\_vel} genannt, berechnet. Danach
werden damit die zusätzlich wirkenden Kräfte verrechnet.
\subsubsection{Externe Kräfte}
\label{sec:org77e0a21}
Mit der Impulserhaltung lässt sich lediglich die Geschwindigkeit oder Masse
eines Objektes berechnen. Allerdings hat man dann noch keine Vorstellung von der
Beschleunigung die auf ein Objekt wirkt. Allerdings stellt sich heraus, dass ein
einfacher Trick verwendet werden kann, um diese Berechnung zu vereinfachen. Die
Beschleunigung lässt sich wie folgt mathematisch berechnen: \[\vec{a} =
\frac{\triangle v}{\triangle t}\] Da die Zeit im hier verwendeten
Schrittverfahren pro Schritt immer \texttt{1} beträgt, kann man einfacher sagen:
\[\vec{a} = \triangle v}\]
\begin{enumerate}
\item \(\triangle t\) erlaubte es dann einfach, die extern wirkenden Kräfte, wie
beispielsweise die Schwerkraft davon abzuziehen. \[a_neu = \trangle t -
   9.81\]. Die neue Beschleunigung lässt sich dann wieder gleich in eine
Geschwindigkeit umrechnen und zur ursprünglichen Geschwindigkeit addieren, um
die neue Geschwindigkeit zu erhalten.
\item Der Luftwiderstand war einiges komplizierter zu berechnen als die
Schwerkraft. Zum einen mussten technische Probleme und Limitierungen
berücksichtigt werden, wie beispielsweise die fehlende Präzision bei sehr
kleinen Dezimalzahlen. Auch die fehlenden Daten und Werte stellten sich als
grosses Problem heraus. Aus Zeitgründen wurden verschiedene Quellen gesammelt
und verrechnet, anstatt die tatsächlichen Werte zu finden oder zu berechnen.
\begin{enumerate}
\item Tatsächlich gab es überraschend wenig Daten über die Form oder \texttt{c-Werte} von
Raketen. Also wurde am Ende einfach der eines \href{http://www.staedtisches-gymnasium-wermelskirchen.de/sites/default/files/physik/Fall-Papierkegel-mit-Luftwiderstand.pdf}{Kegels} angenommen, also
\texttt{0.75}. Obwohl dies bei weitem nicht Perfekt ist, ähneln die meisten Raketen
in ihrer Form einem Kegel doch sehr, wodurch hier nur geringe Fehler
entstehen sollten. In einer späteren Version des Simulators soll auch
dieser Wert in den Raketeneinstellungen bestimmbar sein. Da sich der Wert
aber bei jeder einzelnen Stufe ändert und bei der \texttt{Saturn V} sogar während
einer Stufe nicht konstant bleibt, war es nicht möglich, diese Funktion in
absehbarer Zeit zu implementieren, wodurch \texttt{0.75} als Konstante gesetzt
wurde.
\item Auch bei der Berechnung des Luftdrucks kamen neue Probleme auf. Neben den
ursprünglichen Problemen mit der korrekten Implementierung der Formel, gab
es auch seltsame Fehler mit \texttt{Golang}. So musste am Ende für jede Höhe über
\texttt{100km} (\href{https://de.wikipedia.org/wiki/K\%C3\%A1rm\%C3\%A1n-Linie}{Karman-Linie}) der Luftdruck auf \texttt{0} gesetzt werden, da sonst die
Werte nicht mehr zu verarbeiten gewesen.
\item Mit der Formel für den Luftwiderstand
\[F_l = \frac{1}{2} * A * c_w * p * v²\]
lässt sich die aktuelle Kraft des Luftwiderstands berechnen. Diese muss
dann allerdings noch durch die Masse der Rakete geteilt werden, um daraus
eine Beschleunigung zu machen, welche dann wie oben beschrieben von der
Geschwindigkeit abgezogen werden kann.
\end{enumerate}
\end{enumerate}
\subsection{Raketen}
\label{sec:orgcaf2288}
Zwar waren bereits gute Werte für die \texttt{Saturn V} vorhanden, aber es stellte sich
als überraschend kompliziert heraus gute Daten für die \texttt{Falcon 9} oder andere
Raketen zu finden. Zum einen liegt dies an der Tatsache, dass \texttt{SpaceX} eine
private Firma ist, welche natürlich nicht ihre gesamten Werte öffentlich macht,
zum anderen liegt es aber ebenfalls an der \texttt{US-Regierung}, die den öffentlichen
Zugang zu solchen Informationen erschwert, da diese oftmals als relevant für die
nationale Sicherheit angesehen werden. Daher mussten für \texttt{Falcon 9} einige
Annahmen und Schätzungen getroffen werden. Die Mehrheit der Daten stammten aber
ursprünglich aus diesem inoffiziellen \href{https://www.reddit.com/r/spacex/comments/3lsm0q/f9ft\_vs\_f9v11\_fuel\_mass\_flow\_rate\_isp/}{Reddit Post} und ergeben tatsächlich
Flugdaten, welche der echten Rakete ähneln.
\subsection{Plotting}
\label{sec:orge608555}
Die ursprüngliche Planung unserer Software enthielt die Absicht, die
berechneten Flugdaten als Graphen zu exportieren. Aufgrund von zeitlichen und
technischen Limitierungen mussten diese Funktionen allerdings weggelassen
werden, sollen aber in einer späteren Version der Software implementiert werden.
Aktuell werden die Flugdaten lediglich in der Konsole, sowie einem \texttt{CSV-ähnlichen}
Format exportiert und gespeichert.
\section{Resultate}
\label{sec:org6175036}
TODO.
\end{document}
